\documentclass[12pt]{article}
%compile using XeLaTeX
\usepackage{diary_style}
\setlength{\footskip}{\paperheight
  -(0.75in+\voffset+\topmargin+\headheight+\headsep+\textheight)
  -0.75in}

%\fontspec{Times New Roman}

\DeclareMathOperator{\di}{d\!}
\newcommand*\Eval[3]{\left.#1\right\rvert_{#2}^{#3}}

\begin{document}

%\doublespacing
\vspace{1.0 \baselineskip}

\begin{flushright}
	Alexander Caines\\
	MATH 301\\
	1-30-2021\\
\end{flushright}

\begin{center}
	\textbf{\underline{HMWK 1}}
\end{center}


%\vspace{0.5 \baselineskip}

\begin{enumerate}
	\item[1.]
		\begin{enumerate}
			\item[(a)] Since the calculated t-statistic, 4.81, is greater than the calculated critical value 
				of 4.32, the null hypothesis---that the concentration in the sampled region exceeds the stated
				 background value---is accepted. The result does not surprise me because 45.31 is much greater than 20.
				%The observed average 45.31 is greater than the actual (20) by more than 5.26.
				%Furthermore, after carrying out a significance test at 0.01, the 
				%z score was found to be 8.34, which falls into the rejection region as it 
				%is greater than 7.76. In the end, the result does not surprise me.
			\item[(b)] Given that the p-value is 0.99, I would say that it is extremely likely.
		\end{enumerate}
	\item[* 2.]
		\begin{enumerate}
			\item[(a)] Despite the fact that the distribution 
				is not normal (the proposed mean is nowhere near the median fo the range provided),
				because there are sufficiently 
				many entries in the population, the central 
				limit theorem allows us to test the hypothesis 
				about the value of the population mean consumption.

			\item[(b)] Upon running a t-test, under the following hypothesis $$H_0: \mu<200, H_a: \mu > 200$$
				did not fall under the critical region. So the null is rejected, implying that the mean consumption 
				was not at most 200mg.
		\end{enumerate}
	\item[3.] No, the data does not suggest that the condition has not 
		been met as the average is within the critical region.
	\item[4.]
		\begin{enumerate}
			\item[(a)] The relevant test statistic in this case would be testing for proportion. 
				Consider the following hypothesis: $$H_0: p\hat{}=p_0 $$
				$$H_a: p_0 > \frac{1}{75} \text{ or } \frac{1}{75} < -p\hat{}$$
				It can be concluded that the rate of teh chromosome defect in question 
				differs from the presumed rate. A Type 1 Error could have been made when arriving at the conclusion.

			\item[(b)] The pnorm calculated came out to be 0.089. So, 
				it would have been rejected at a significance level of 0.2.
		\end{enumerate}
	\item[5.] Since the calculated zscore was 0.74 and the associated pnorm was 0.77, both of which are greater than
		the expected proportion, 0.10, it can be concluded that more than 10\% of the population has abstained from alcohol use.
\end{enumerate}
\end{document}
