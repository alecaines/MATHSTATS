\documentclass[12pt]{article}
%compile using XeLaTeX
\usepackage{diary_style}
\setlength{\footskip}{\paperheight
  -(0.75in+\voffset+\topmargin+\headheight+\headsep+\textheight)
  -0.75in}

%\fontspec{Times New Roman}

\DeclareMathOperator{\di}{d\!}
\newcommand*\Eval[3]{\left.#1\right\rvert_{#2}^{#3}}

\begin{document}

%\doublespacing
\vspace{1.0 \baselineskip}

\begin{flushright}
	Alexander Caines\\
	MATH-310\\
	3-23-2021\\
\end{flushright}

\begin{center}
	\textbf{\underline{HMWK 3}}
\end{center}


%\vspace{0.5 \baselineskip}

\begin{enumerate}
	\item[(2)] Because none of the elements in the F-Stat column are less than 0.05, it can be conclulded that there 
	is no significance in effects between the fixed factors listed in the problem on the strength of paper. Find the ANOVA table below.\\
		\begin{center}
			\begin{tabular}{c|c|c|c|c}
				& SS & dof & msq & f-stat\\
				A & 6.94 & 1 & 6.94 & 2.97\\
				B & 3.61 & 3 & 1.20 & 0.51\\
				C & 12.33 & 2 & 6.17 & 0.38\\
				AB & 4.05 & 1 & 4.03 & 4.73\\
				Error & 14.04 & 6 & 2.34 & \\
				\hline
				Total & 40.7 & 13 & 20.68 & \\
			\end{tabular}
		\end{center}
	\item[(4)]
\end{enumerate}

\end{document}
